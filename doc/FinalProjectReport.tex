\documentclass[12pt]{article}

\usepackage[utf8]{inputenc}
\usepackage[margin=1in]{geometry}
\renewcommand{\baselinestretch}{1}
\usepackage{indentfirst}

\usepackage{amsmath, amssymb}

\usepackage{hyperref}
\usepackage{cleveref}
\usepackage{graphicx}
\usepackage{float}
\graphicspath{{./figs/}}

\usepackage{natbib}
\bibliographystyle{aasjournal}

\begin{document}

\begin{center}\begin{LARGE}
\textbf{ASTR 5900 Final Project: Hydro1D}
\end{LARGE}\end{center}

\begin{center}
\textbf{Anthony Burrow and Sarah Stangl}
\end{center}

\section{Introduction and Goals}

This project is aimed to model the infall and subsequent shockwave exhibited by
core-collapse supernovae (CCSNe). In modern studies of all types of SNe
(including Type Ia SNe, CCSNe, etc.), one of the most prominent goals involved
is to achieve a better understanding of the so-called progenitor problem. To
briefly summarize, we wish for the ability to understand the different possible
scenarios of a supernova progenitor as well as to diagnose the scenario with
observable features from the subsequent explosion. If this could be achieved,
this would provide insight into stellar evolution as a whole and minimize a
large source of uncertainty in many interesting predictions in astronomy made
by using these SNe.

The goal of this project is only to model the dynamics of the collapse of a
massive ($\sim10\ M_\odot$) star. This is done using a 1D Lagrangian-frame
hydrodynamical code that solves equations of mass conservation, momentum
conservation, and energy conservation, with the addition of an equation of
state. Our secondary and hopeful goal is to find a distinguishable shock event
that should occur when a bulk of infalling material rapidly rebounds off of a
highly dense and degenerate core of neutrons. This shockwave is what signifies
that a CCSN occurs.

\section{Methods}

To generate the aforementioned model, we first assume a star of mass
$10\ M_\odot$ coming directly from hydrostatic equilibrium. The star's collapse
would be caused by a sudden decrease in pressure. The model is separated into a
set number of zones, all with equal mass. Because this is a Lagrangian
hydrocode, the zones always have the same constant mass.

The initial conditions (radius, density, pressure) are all solved for
hydrostatic equilibrium through the Lane-Emden equation. In order to solve
this equation we assume the star---which will be treated as a fully degenerate
Fermi gas until nuclear degeneracy---can be modeled as an $n=3$ polytrope. By
doing this, we therefore assume an equation of state of
$P = K_{4/3}\ \rho^{4/3}$, where $P$ is pressure, $\rho$ is density, and
$K_{4/3} = 1.2 \times 10^{15}$ in CGS units \citep{arnett66}. This equation of
state is for densities under nuclear degeneracy
$\rho_{nuc} = 2.3 \times 10^{14}$ g cm$^{-3}$. We also assume an intial
central density of $\rho_c = 1.0 \times 10^7$ g cm$^{-3}$.

% Insert initial density vs m_int plot

Once these initial conditions have been set, the dynamics of density, pressure,
and velocities are calculated by following the exact difference equations found
in the Appendix of \citet{arnett66}. However, because of time constraints, we
simplify the calculation heavily by not accounting for radiation or neutrinos
as a source of energy. Because we assume a polytropic equation of state, our
model also has zero temperature, and no temperature calculations are needed.
From the initial conditions, we do 10,000 iterations of calculating zone
boundary velocities ($U_j$), boundary positions ($R_j$), specific volume of
each zone ($V_{j + 1/2} = 1 / \rho_{j + 1/2}$), and zone pressure
($P_{j + 1/2}$) in that order. These are calculated using simple difference
equations that describe the equations of mass, momentum, and energy
conservation, and they are all listed in the Appendix of \citet{arnett66}.

From an initial time step (we typically us $\Delta t^0 = 0.001$ second), each
iteration calculates a new maximum time step that holds causality due to sound
velocity. This is done in the same way as \citet{colgatewhite64} by finding the
maximum value (for any $j^\text{th}$ zone) of
$$
\begin{aligned}
\Delta t^{n + 1/2}
&= \frac{0.02 \cdot V^n_{j + 1/2} \Delta t^{n - 1/2}}
        {|V^n_{j + 1/2} - V^n_{j - 1/2}|}, \\
\Delta t^n
&= \frac{1}{2} (\Delta t^{n + 1/2} + \Delta t^{n - 1/2}).
\end{aligned}
$$
We found that this keeps causality well for our purposes; in other words, this
means no negative densities, etc. appear due to over-correction to the
boundary positions.

To reiterate, the pressure we assume is due solely to electron pressure, i.e.
$$
\begin{aligned}
P_j = P(e^-)_j = K_{4/3}\ (1 / V_j)^{4/3}.
\end{aligned}
$$
However, to model neutron degeneracy pressure, for any zone with densities
above nuclear density $\rho_{nuc}$, this pressure is ammended to
$$
\begin{aligned}
P_j = P(e^-)_j + K_3\ (1 / V_j)^3,
\end{aligned}
$$
where
$$
\begin{aligned}
K_3 = \frac{K_0}{27 \rho_c^2 m_n}
\end{aligned}
$$
with $K_0 = 140$ MeV and $m_n = 1.674920 \times 10^{-24}$ g is the mass of a
neutron. This is based on the cold nuclear pressure suggested by and
$K_0 = K_0(0.33)$ found in \citet{bck85} with $\gamma = 3$.

Finally, to cause a collapse, we calculate our initial hydrostatic condition
with $K'_{4/3} = 1.1 \cdot K_{4/3}$ within the equation of state, and then
lower the pressure by using $K_{4/3}$ in our subsequent iterations instead. The
lowered pressure should lead to gravitational domination and a collapse.


\section{Results}

* check hydrostatic equilibrium (no K change)

* check P=0 case (freefall)

* collapse case (K changes by -10%) -> shock?

\section{Conclusion}



\bibliography{FinalProjectReport}

\end{document}
