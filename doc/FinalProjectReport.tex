\documentclass[12pt]{article}

\usepackage[utf8]{inputenc}
\usepackage[margin=1in]{geometry}
\renewcommand{\baselinestretch}{1}
\usepackage{indentfirst}

\usepackage{amsmath, amssymb}

\usepackage{hyperref}
\usepackage{cleveref}
\usepackage{graphicx}
\usepackage{float}
\graphicspath{{./figs/}}

\begin{document}

\begin{center}\begin{LARGE}
\textbf{ASTR 5900 Final Project: Hydro1D}
\end{LARGE}\end{center}

\section{Introduction and Goals}

This project is aimed to model the infall and subsequent shockwave exhibited by
core-collapse supernovae (CCSNe). In modern studies of all types of SNe
(including Type Ia SNe, CCSNe, etc.), one of the most prominent goals involved
is to attain more understanding of the so-called progenitor problem.

\section{Methods}

* assumptions (mass, radius)

* hydrostatic LE eq. solved for initial density using aforementioned assumptions

* hydrocode details

\section{Results}

* check hydrostatic equilibrium (no K change)

* check P=0 case (freefall)

* collapse case (K changes by -10%) -> shock?

\section{Conclusion}




\end{document}
