\documentclass[12pt]{article}

\usepackage[utf8]{inputenc}
\usepackage[margin=1in]{geometry}
\renewcommand{\baselinestretch}{1}
\usepackage{indentfirst}

\usepackage{amsmath, amssymb}

\usepackage{hyperref}
\usepackage{cleveref}
\usepackage{graphicx}
\usepackage{float}
\graphicspath{{./figs/}}

\usepackage{natbib}
\bibliographystyle{aasjournal}

\begin{document}

\begin{center}\begin{LARGE}
\textbf{Final Project Topic}
\end{LARGE}\end{center}


\section*{}

For this project, I'll be working with Sarah Stangl.

\vspace{5mm}

At the bare minimum, this project will involve making a simple 1D (Lagrangian)
hydrocode that will model velocities, temperatures, pressures, etc. within a
star as a function of time. This code should also be able to introduce a
(simple) shock event, which could be useful in creating models of core-collapse
SNe.

To do this numerically, we will first either use a piecewise-constant or more
likely a piecewise-linear method to approximate the interface states that are
needed for calculation. If time permits, this method could be possibly upgraded
to the more accurate piecewise-quadratic method, and the results may be
compared. Initially we may likely start designing this code run in serial, but
again as time permits this code may be parallelized if possible. It may even be
necessary.

I've already looked at the appendix of the \citet{arnett66} paper that covers a
lot of steps involved in creating a basic hydrocode. It is a bit different
because it is a very old paper, but I believe it uses the piecewise-linear
method (averages) to get interface values.

There was also another great source that was shown to me by
Michael Zingale\footnote{http://zingale.github.io/hydro1d/} that has a lot of
example code and a detailed book for astrophysics students on writing a
hydrocode.

\vspace{5mm}

This project interests me not only because it easily relates to my research in
SNe, but because there are many different ways to optimize a hydrocode. Code
optimization is a skill that I am always trying to better for myself, and this
is the perfect way to practice this and gain a lot of knowledge about what
goes on in the background of creating models related to my field of research.


\bibliography{topic}

\end{document}
